\chapter{Analýza vyučovania}
\label{chap:analyza_vyucovania}

Virtuálne sieťové laborarórium má byť nasadené na vybrané predmety vyučované na katedre. Na to treba analyzovať vyučované témy týchto predmetov. Na základe toho sa budú získavať a testovať zariadenia, čomu je venovaná kapitola \ref{chap:virt_zariadenia} - \nameref{chap:virt_zariadenia}.

V tejto kapitole opisujem vyučované témy týchto predmetov:

\begin{itemize}[noitemsep]
    \item Počítačové siete 1 (5BN103)
    \item Počítačové siete 2 (5BN104)
    \item Projektovanie sietí 1 (5IN116)
    \item Projektovanie sietí 2 (5IP111)
    \item CCNA Security
    \item Pokročilé prepínanie v informačno-komunikačných sieťach (5IN139)
    \item Pokročilé smerovanie v informačno-komunikačných sieťach (5IN124)
\end{itemize}

Výber predmetov ovplyvňoval fakt, že na nich vyučujú sieťové technológie. Nástroj má byť v prvom rade používaný na predmetoch, kde sa vyučujú pokročilejšie sieťové technológie t.j. Projektovanie sietí 1, Projektovanie sietí 2, Pokročilé prepínanie v informačno-komunikačných sieťach a Pokročilé smerovanie v informačno-komunikačných sieťach.

V nasledujúcich častiach budú opísané zariadenia, ktoré sa používajú pri výučbe týchto predmetov, ako aj vyučované technológie.

Vyučované technológie boli získané z informačných listov predmetov a z plánov predmetu od vyučujúcich. Zoznam vyučovaných technológii je dostupný v kapitole \ref{chap:cd} v bodoch \ref{item:zoznam_technologii_s_podporou_zariadeni} a \ref{item:zoznam_technologii_txt}




\section{Počítačové siete 1}

Predmet obsahuje témy z oblasti CCNA 2 vrátane prepínacích technológii z CCNA 3. Momentálne sa na predmete používa iba nástroj Packet Tracer.

Na predmete sa vyučuje IPv4 a IPv6 statické smerovanie, RIPv2, RIPng, SVI, STP BPDU Guard, PortFast, VLAN, VLAN Trunk 802.1Q, InterVLAN smerovanie - Router on a Stick, VTP v1/v2/v3, STP, PVST+, RPVST+, Extended VLAN, L2 EtherChannel PAgP a LACP, L3 EtherChannel PAgP a LACP, HSRP IPv4, HSRPv2 IPv4 a IPv6, VRRPv2 IPv4 VRRPv3 IPv4 a IPv6, GLBP IPv4 a IPv6, ACL IPv4 a IPv6, DHCP IPv4 a IPv6, NAT, LLDP, CDP, Syslog, NTP, SNMP, SPAN.

Na predmete sa pracuje predovšetkým so smerovačmi a prepínačmi Cisco a jednoduchými koncovými zariadeniami v rámci možností nástroja Packet Tracer. V budúcnosti sa uvažuje o integrácii Juniper smerovačov a pokročilejších koncových zariadení na platforme Linux a Windows.




\section{Počítačové siete 2}

Predmet obsahuje témy z oblasti CCNA 3 a CCNA 4 okrem prepínacích technológii. Momentálne sa na predmete používajú nástroje Packet Tracer a na niektoré topológie nástroj Dynamips/\\Dynagen. V druhom menovanom nástroji topológie pozostávajú z zariadení Cisco 2691.

Na predmete sa vyučujú témy EIGRP IPv4 a IPv6, OSPFv2 Single-Area a Multi-Area, OSPFv3 Single-Area a Multi-Area, PPP, MLPPP, HDLC, PPPoE, GRE, eBGP IPv4.

Na predmete sa pracuje predovšetkým so smerovačmi a prepínačmi Cisco a jednoduchými koncovými zariadeniami v rámci možností nástroja Packet Tracer. V budúcnosti sa uvažuje o integrácii Juniper smerovačov a pokročilejších koncových zariadení na platforme Linux a Windows.




\section{Projektovanie sietí 1}

Predmet obsahuje niektoré témy z oblasti CCNP Routing a ďalších pokročilých smerovacích technológii. Momentálne sa na predmete používa nástroj Dynamips/Dynalab. V ňom pozostávajú topológie zo zariadení Cisco 2691 a Cisco 7200.

Na predmete sa vyučujú témy OSPFv2 Multi-Area, IS-IS IPv4, IGMP v1/v2/v3, IGMP Snooping, PIM Dense Mode/Sparse Mode/Sparse-Dense Mode, PIM Any-Source Multicast, PIM Source-Specific Multicast, Manual RP, Auto-RP, BSR, Anycast RP, BGP IPv4, Router Reflector, MP-BGP,BGP mVPN, Hub \& Spoke VPN, Draft Rosen, BGP L3 VPN, MPLS, LDP, RSVP, VPLS.

Na predmete sa pracuje predovšetkým so smerovačmi a prepínačmi Cisco. Koncové zariadenia sa takmer vôbec nepoužívajú, iba ak by sa skupina rozhodla pracovať s fyzickými zariadeniami. V budúcnosti sa uvažuje o integrácii Juniper smerovačov a pokročilejších koncových zariadení na platforme Linux a Windows, hlavne pre účely vyučovania \emph{multicast} technológii.



\section{Projektovanie sietí 2}

Predmet obsahuje témy z oblasti pokročilých smerovacích technológii. Výučba tohto predmetu bola v šk. roku 2017/2018 realizovaná v nástroji EVE-ng v rámci pilotného nasadenia do vyučovania.

Na predmete sa vyučujú témy VPLS, EVPN, Seamless MPLS, BGP mVPN NG.


Na predmete sa pracovalo so smerovačmi a prepínačmi Cisco a smerovačmi Juniper a Nokia. Nástroj EVE-ng podporuje koncové zariadenia na platforme Linux a Windows a je ich možné integrovať do topológie.




\section{CCNA Security}
  
Predmet obsahuje prehľad tém a technológii z oblasti bezpečnosti v rámci linkovej, sieťovej a aplikačnej vrstvy. Výučba tohto predmetu je plánovaná na šk. roku 2018/2019 namiesto predmetu Optimalizácia konvergovaných sietí. Zvažuje sa nad jeho realizáciou v nástroji EVE-ng v rámci ďalšieho nasadenia do vyučovania.

Keďže predmet je nový a jeho osnova ešte nie je pevne stanovená, zoznam vyučovaných technológii nie je uvedený. Prehľad tém vyučovaných na kurze čerpá z materiálov Cisco Network Security (IINS) (210-260), ktorý je dostupný na stránke \cite{ccna_security_topics}.

Predmet vyžaduje Cisco zariadenia, konkrétne prepínače, smerovače, popr. Cisco firewall a koncové zariadenia na platforme Linux alebo Windows.





\section{Pokročilé prepínanie v informačno-komunikačných sieťach}

Predmet obsahuje témy z oblasti CCNP Switching. Momentálne sa na predmete používajú fyzické zariadenia, keďže katedra momentálne nedisponuje riešením na virtualizáciu prepínačov.

Vyučované témy na tomto predmete sa do veľkej miery zhodujú s predmetom Počítačové siete 1, avšak témy sú preberané podrobnejšie. Osnova predmetu obsahuje navyše témy IP SLA, STP BPDU Filter, MST, CEF, MLS, FHRP IPv4 a IPv6, NTP Authentication, Cisco ISL trunks, DHCP Snooping, PVLAN.

Na predmete sa pracuje predovšetkým s fyzickými prepínačmi a Cisco. Nástroj EVE-ng umožňuje do topológie integrovať aj rôzne Cisco prepínače.





\section{Pokročilé smerovanie v informačno-komunikačných sieťach}

Predmet obsahuje témy z oblasti CCNP Routing. Momentálne sa na predmete používa nástroj Dynampis/Dynalab. Ten obsahuje topológie, ktoré využívajú smerovač rady Cisco 7200, keďže je to jedniný Dynamips smerovač, ktorý v plnom rozsahu podporuje technológie vyučované na predmete.

Vyučované témy na tomto predmete sa do veľkej miery zhodujú s predmetom Počítačové siete 2, avšak témy sú preberané podrobnejšie. Osnova predmetu obsahuje navyše témy PBR, Route redistribution, Route filtering, IP SLA, MP-BGP.

Na predmete sa pracuje predovšetkým so smerovačmi a prepínačmi Cisco. Nástroj EVE-ng umožňuje do topológie integrovať aj rôzne Cisco smerovače, od jednoduchších, až po pokročilejšie.