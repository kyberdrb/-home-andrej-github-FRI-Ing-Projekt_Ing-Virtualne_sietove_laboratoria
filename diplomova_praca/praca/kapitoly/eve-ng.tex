\chapter{EVE-ng}
\label{chap:eve_ng}

EVE-ng je virtuálne sieťové laboratórium skladajúcej sa zo serverovej časti postavenej na platforme Linux a klientskej časti, ktorú tvorí webová aplikácia, ako je už spomenuté v kapitole \ref{chap:virt_lab_eve_ng}.

Serverová časť je realizovaná ako tzv. LAMP server, podobne ako nástroj ViRo2. LAMP je skratka pre \emph{Linux Apache MySQL PHP}. Webový server Apache poskytuje webovú stránku. MySQL je relačná databáza ukladajúca informácie o používateľoch webového rozhrania EVE-ng. Správe používateľských účtov sa venuje kapitola \ref{chap:sprava_pouzivatelov}. PHP je použitý na spracovanie REST API volaní z webového rozhrania. Webové rozhranie je vytvorené pomocou nástrojov \emph{Twitter Bootstrap v3.3.6} a \emph{AngularJS v1.5.6}.

V tejto kapitole bude opísaný proces nasadzovania EVE-ng servera do sieťovej infraštruktúry katedry: jeho inštalácie, následnej úpravy a základnej administrácie EVE-ng servera. Všetky kroky sú podrobne opísané v adresári s návodmi pre EVE-ng v kapitole \ref{chap:cd} v bode \ref{item:prilohy_cd_eve_ng_adresar}.





\section{Inštalácia}
\label{chap:eve_ng_instalacia}

EVE-ng bol inštalovaný a testovaný na dvoch platformách:

\begin{itemize}[noitemsep]
    \item VMware Workstation Player
    \item Fyzický server
\end{itemize}

Parametre VMware virtuálneho stroja a fyzického servera sú uvedené v tabuľke \ref{tab:server_parameters}. V oboch prípadoch bol EVE-ng server nasadený do DMZ zóny, preto v riadku \emph{IP adresa} uvádzam len posledný oktet ich IPv4 adries, keďže adresný rozsah DMZ zóny je na katedre známy.

\begin{longtabu} to \textwidth {| X[5.0,cm] | X[5.0,cm] | X[5.0,cm] |}
\caption{Parametre EVE-ng serverov}
\label{tab:server_parameters} \\
\hline
    \textbf{Parametre \textbackslash~ Server} & \textbf{VMware} & \textbf{Fyzický server} \\
\hline
    \textbf{CPU} & 16 & 8 \\
\hline
    \textbf{Operačná pamäť (GB)} & 64 & 48 \\
\hline
    \textbf{EVE-ng verzia} & 2.0.3-80 & 2.0.3-86 \\
\hline
    \textbf{IP adresa} & .49 & .50 \\
\hline
\end{longtabu}

\noindent
Uvedené tvrdenia platia pre EVE-ng vo vydaní \emph{Community Edition} vo verzii 2.0.3-86.

\noindent
Postup inštalácie EVE-ng servera môžeme zhrnúť do týchto krokov, ktoré sú stručne zhrnuté v nižšie uvedenom zozname, pričom každý z krokov bude vysvetlený v nasledujúcich odstavcoch.
\begin{enumerate}[noitemsep]
    \item \label{item:vzdialena_pracovna_plocha} Vytvorenie vzdialenej pracovnej plochy a VMware Workstation (kap. \ref{chap:cd}, body \ref{item:vnc1}-\ref{item:vmware_extra_instalacia})
    \item \label{item:ubuntu_install} Inštalácia Ubuntu Server 16.03.4 LTS (kap. \ref{chap:cd}, body \ref{item:instalacia_ubuntu_a_eve_ng}, \ref{item:aktualizacia_eve_ng})
    \item \label{item:ubuntu_config} Konfigurácia Ubuntu Server (kap. \ref{chap:cd}, bod \ref{item:instalacia_ubuntu_a_eve_ng})
    \item \label{item:eve_ng_install_on_ubuntu} Inštalácia EVE-ng do Ubuntu Server (kap. \ref{chap:cd}, bod \ref{item:instalacia_A_konfiguracia_eve_ng})
    \item Konfigurácia EVE-ng servera\\

\noindent
Konfigurácia operačného systému Ubuntu a EVE-ng servera sa nevykonala automaticky. Po inštalácii nástroja EVE-ng mal server mnohé nedostatky, ktoré bolo potrebné ošetriť. Preto sme vytvorili návody a skripty, ktoré uľahčujú konfiguráciu a administráciu servera a budú popísané v ďalších častiach tejto kapitoly. Konfigurácia EVE-ng servera zahŕňala:


    \item \label{item:obnovenie_suborov} Obnovenie (konfiguračných) súborov, adresárov a virtuálnych zariadení zo zálohy (kap. \ref{chap:cd}, bod \ref{item:obnova_zalohovanie})
    \begin{itemize}[noitemsep]
        \item Skripty (kap. \ref{chap:cd}, bod \ref{item:skripty})
        \item Zariadenia (kap. \ref{chap:cd}, body \ref{item:pridavanie_konverzia_zariadeni}-\ref{item:pridavanie_win})
        \item Databázy (kap. \ref{chap:cd}, bod \ref{item:obnova_zalohovanie})
    \end{itemize}
    \item \label{item:automatizacia_zalohovania} Automatizácia zálohovania kritických súborov, adresárov a databáz (kap. \ref{chap:cd}, bod \ref{item:obnova_zalohovanie})
    \item \label{item:iol_license_generate} Pridanie Cisco IOL/IOU licencie (kap. \ref{chap:cd}, bod \ref{item:iou_licencia})
    \item \label{item:zabezpecenie_servera} Zabezpečenie servera (kap. \ref{chap:cd}, bod \ref{item:zabezpecenie})
    \begin{itemize}[noitemsep]
        \item Systém
        \item SSH
        \item Webový server
    \end{itemize}
    \item \label{item:uprava_sablon_zariadeni} Úprava šablón (kap. \ref{chap:cd}, bod \ref{item:uprava_sablon})
    \item \label{item:uprava_zdrojovych_kodov} Úprava zdrojových kódov (kap. \ref{chap:cd}, body \ref{item:spristupnenie_pouzivatelskych_roli}-\ref{item:zatvorenie_topologie_so_spustenymi_zariadeniami})
\end{enumerate}

Inštalačný proces pre obe platformy (body \ref{item:ubuntu_install}, \ref{item:ubuntu_config}, \ref{item:eve_ng_install_on_ubuntu}), virtuálnu aj fyzickú, bol takmer zhodný, líšil sa iba v úvodných krokoch. Pri inštalácii pre VMware bolo totiž potrebné na server, na ktorom bol VMware nainštalovaný, pridať VNC prístup na hosťovský operačný systém (bod \ref{item:vzdialena_pracovna_plocha}) a doplniť grafické prostredie, aby bolo možné ovládať grafické rozhranie VMware Player a spustiť virtuálny stroj, keďže VMware Player dokáže vytvoriť a spustiť virtuálny stroj iba z grafického rozhrania.

Rozdielov medzi oboma inštaláciami je niekoľko. Prvým z nich je už spomenutá verzia. VMware inštalácia má nižšiu verziu, pretože bola nainštalovaná skôr. VMware inštalácia slúžila na prvotné odladenie a pilotné nasadenie do vyučovania. Nebola v nej vykonaná takmer žiadna dodatočná konfigurácia, okrem importu zariadení pre topológie.

Následná inštalácia EVE-ng na fyzický server vychádzala zo skúseností získaných z inštalácie EVE-ng do VMware prostredia. EVE-ng fyzický server bol odladený a do veľkej miery testovaný. Testovaniu zariadení v EVE-ng sa venujem v kapitole \ref{chap:testovanie_zariadeni} - \nameref{chap:testovanie_zariadeni}.

Obnovovanie súborov a adresárov (bod \ref{item:obnovenie_suborov}) slúži na rýchle obnovenie upravených konfiguračných súborov, adresárov, databázy a virtuálnych zariadení zo zálohovacieho servera na EVE-ng server. Obnovu týchto súborov musíme vykonať manuálne, avšak ich zálohovanie je automatizované (bod \ref{item:automatizacia_zalohovania}). Tento krok môžeme preskočiť, ak predtým ešte nebola vytvorená záloha príslušným zálohovacím skriptom. Zálohovanie je vysvetlené v časti \ref{chap:zalohovanie}.

Pridanie Cisco IOL/IOU licencie (bod \ref{item:iol_license_generate}) je dôležitým krokom, bez ktorého by sme neboli schopní spúšťať Cisco IOL/IOU zariadenia. Pre vygenerovanie tejto licencie bol použitý skript, ktorý sme pre tento účel vytvorili. Skript po spustení automatický vygeneruje IOL licenciu a vloží ju do správneho adresára či už na EVE-ng, alebo na GNS3 serveri, keďže obidva nástroje sú podporované a schopné spúšťať IOL zariadenia. Význam týchto zariadení bude vysvetlený v kapitole \ref{chap:testovanie_technologii} - \nameref{chap:testovanie_technologii} v časti \nameref{chap:testovanie_technologii_vyhodnotenie}.

Zabezpečenie servera \ref{item:zabezpecenie_servera} spočívalo hlavne v zabezpečení operačného systému, SSH prístupu a webového servera Apache. Zabezpečenie operačného systému obsahovalo vytvorenie štandardného používateľského konta so \emph{sudo} oprávneniami. Ten sa bude používať namiesto \emph{root} používateľa, čím bude zaistená vyššia bezpečnosť pri používaní systému. Štandardný používateľský účet sa využíva aj na odchytávanie prevádzky z topológie namiesto \emph{root} používateľa, ako bude ozrejmené v kapitole \ref{chap:pouzivanie_eve_ng} v bode \ref{item:vzdialeny_pristup}.

Zabezpečenie SSH prístupu zahŕňalo zablokovanie \emph{root} používateľa, explicitné definovanie povolených používateľov a skupín, vygenerovanie SSH kľúčov a vypnutie autentifikácie heslom. Autentifikácia SSH kľúčmi má aj tú výhodu, že oproti autentifikácii heslom nie je nutné zadávať heslo, čím odpadá aj nutnosť pamätať si ho. Každý počítač, ktorý by chcel pristupovať k EVE-ng serveru, by si musel svoj verejný SSH kľúč nahrať na server k danému používateľskému účtu. Avšak kvôli jednoduchosti bola ponechaná autentifikácia heslom pre oba servery.

Zabezpečenie webového servera Apache sa skladalo z vygenerovania SSL certifikátu, aktivácie protokolu HTTPS a presmerovania požiadaviek z HTTP na protokol HTTPS. Webový server však nie je zabezpečený na ani jednom serveri, pretože bolo potrebné odchytávať nezašifrované správy vymieňané prostredníctvom REST API medzi klientom a serverom. Analýza týchto vymienaných správ pomohla upraviť funkcie v nástroji EVE-ng, ktoré je bližšie opísané v časti \ref{chap:eve_ng_uprava_zdroj_kodov}.

Na serveri bola po pridaní virtuálnych zariadení vykonaná aj úprava šablón (bod \ref{item:uprava_sablon_zariadeni}) spustením nami vytvoreného skriptu. Používateľ vďaka tejto úprave nemusí pri vytváraní topológie premýšľať nad technickými parametrami zariadenia, ktoré do topológie pridáva. Namiesto toho sa môže sústrediť na vyučovanú problematiku. Tak sa vytváranie topológii aj samotné vyučovanie stane plynulejším. 

Každé zariadenie, ktoré je možné do topológie pridať, si totiž načíta svoje technické parametre zo súboru zvaného šablóna. Tie sú uložené v adresári \texttt{/opt/unetlab/html/templates/}. V šablóne môže byť pre zariadenie definovaný napr. počet pridelených jadier CPU, maximálne množstvo alokovateľnej operačnej pamäte, spúšťacie parametre zariadenia a pod. Úpravy šablón boli vykonané na základe testovania vybraných zariadení, ktoré je opísané v kapitole \ref{chap:testovanie_zariadeni_benchmark} - \nameref{chap:testovanie_zariadeni_benchmark}. Skript na úpravu šablón je súčasťou príloh (kap. \ref{chap:cd}, bod \ref{item:uprava_sablon_skript}).

Úprava zdrojových kódov (bod \ref{item:uprava_zdrojovych_kodov}) bola potrebná na to, aby bolo vylepšené používanie nástroja EVE-ng. Jednotlivé úpravy sú popísané v časti \ref{chap:eve_ng_uprava_zdroj_kodov}.





\section{Úprava nástroja EVE-ng}
\label{chap:eve_ng_uprava_zdroj_kodov}

Po inštalácii EVE-ng servera je odporúčané upraviť aj konkrétne časti jeho zdrojového kódu, aby sa vylepšila jeho funkcionalita. V dôsledku toho sa zlepší použiteľnosť nástroja vo vyučovaní. Medzi tieto úpravy patrí:

\begin{itemize}[noitemsep]
    \item Sprístupnenie používateľských rolí.
    \item Možnosť úpravy používateľských atribútov.
    \item Vypnutie správy o nízkom rozlíšení obrazovky.
    \item Zatvorenie topológie so spustenými zariadeniami.
\end{itemize}

Tieto úpravy vyplynuli z požiadaviek, ktoré boli špecifikované kritériami v úvode kapitoly \ref{chap:porovnavacie_kriteria}. Sprístupnenie používateľských rolí umožní jednotlivým používateľským roliam prideliť činnosti, ktoré sú oprávnení vykonávať vo webovom rozhraní. Úprava používateľských atribútov umožní upravovať a odstraňovať atribúty jednotlivých používateľov z webového rozhrania. Vypnutie správy o nízkom rozlíšení obrazovky umožní mať na počítači otvorenú topológiu v menšom okne, čo je užitočné napr. vtedy, keď chceme mať vedľa otvorenej topológie aj zadanie úlohy v textovom dokumente pre danú topológiu. Zatvorenie topológie 

Takisto je vhodné po inštalácii upraviť aj konfiguráciu klientských počítačov, ktoré budú webové rozhranie EVE-ng používať. V dôsledku týchto úprav bude práca s topológiou pre používateľa pohodlnejšia . Úpravám na klientskej strane sa venujem v bode \ref{item:vzdialeny_pristup} v časti \ref{chap:vytvorenie_topo_eve-ng} - \nameref{chap:vytvorenie_topo_eve-ng}.





\subsection{Metodika}

Na to, aby sme mohli odhadnúť, ktoré časti nástroja EVE-ng treba upraviť, sme použili rôzne nástroje. Hlavná diagnostika bola vykonaná nástrojmi \emph{Wireshark}, \emph{grep} a inšpektorom prvkov stránky vo webovom prehliadači.

Nástroj \emph{Wireshark} slúžil na odchytávanie vymieňaných správ prostredníctvom REST API. Keďže webový server nebol zašifrovaný a používal HTTP protokol, mohli sme skúmať tieto správy v nezašifrovanom texte. Následne sme hľadali číselný kód alebo časť z názvu správy v súboroch vo webovom adresári EVE-ng pomocou nástroja \emph{grep}. Takisto bola použitá tzv. inšpektor prvkov (\emph{Inspector}), ktorý dokázal identifikovať prvky stránky a na základe nich ďalej postupovať. Tieto nástroje umožnili spresniť odhad na tie časti zdrojového kódu, ktorých zmena by s veľkou pravdepodobnosťou mohla vyriešiť problémy uvedené v zozname uvedenom v úvode časti \ref{chap:eve_ng_uprava_zdroj_kodov}.






\subsection{Sprístupnenie používateľských rolí}
\label{chap:eve_ng_pouzivatelske_role}

V EVE-ng Community Edition je pre používateľov dostupná iba jedna používateľská rola \mbox{\emph{admin}}, čiže administrátor. Je to tak preto, lebo \emph{Community} verzia je určená pre osobné použitie, kde sa nepredpokladá viac používateľov, než je používateľ sám. Takéto správanie však nie je vhodné pre nasadenie do vyučovacieho procesu.

Odchytili sme preto správy pri úprave ľubovoľného používateľa. Zistili sme, že sa o.i. posiela aj správa \\
\texttt{Successfully listed user roles (60041)}. Po vyhľadaní výskytu kódu tejto správy v súboroch webového adresára sa ukázalo, že sa vyskytuje aj v súbore \\
\texttt{/opt/unetlab/html/includes/functions.php}.

Na sprístupnenie ďalších používateľských rolí, \emph{editor} - učiteľ a \emph{user} - používateľ resp. študent, bolo potrebné ich odkomentovať z funkcie \texttt{listRoles} v spomenutom súbore.

To vo web rozhraní sprístupnilo v dialógovom okne na vytvorenie a úpravu používateľa ďalšie používateľské role (obrázok \ref{obr:eve_ng_pouzivatelia_dialog}).

\begin{figure}
    \centering
    \includegraphics[width=0.9\textwidth,trim={10cm 1cm 10cm 4cm},clip]{eve_ng_pouzivatelia_dialog}
    \caption{Dialógové okno na vytvorenie a úpravu používateľa}
    \label{obr:eve_ng_pouzivatelia_dialog}
    
    \centering
    \includegraphics[width=0.9\textwidth,trim={8.5cm 6cm 16cm 4.5cm},clip]{eve_ng_pouzivatelia_mysql}
    \caption{Zoznam používateľov v MySQL databázi}
    \label{obr:eve_ng_pouzivatelia_mysql}
\end{figure}

Po vytvorení používateľa s inou rolou než \emph{admin}, napr. \emph{user}, sa vo web rozhraní v zozname používateľov stále zobrazujú ako \emph{admin}, hoci v MySQL databázi sú uložení pod správnou rolou v stĺpci \texttt{role} (obrázok \ref{obr:eve_ng_pouzivatelia_mysql}).

Zachytená komunikácia obsahovala aj záznam so správou \\
\texttt{Successfully listed users (60040)} (obrázok \ref{obr:eve_ng_60040}),
ktorej kód sa nachádzal aj v súbore \\
\texttt{/opt/unetlab/html/includes/api\_uusers.php}.

Riešenie spočívalo v úprave funkcii \texttt{apiGetUUser} a \texttt{apiGetUUsers} v spomenutom súbore. Prvá spomenutá funkcia sa stará o získanie informácii o jednom používateľovi, ďalšia o získanie atribútov všetkých používateľov z MySQL databázy. V oboch funkciách sa však vyskytovala rovnaká chyba, a síce, že používateľská rola sa v príkaze \emph{SELECT} napevno prepisovala na rolu \emph{admin}.

Stačilo v týchto príkazoch prepísať názov používateľskej role z pevnej hodnoty \emph{admin} na názov stĺpca používateľskej role t.j. \texttt{role}.

Po vykonanej úprave sa aj vo webovom zozname používateľov zobrazovala ich správna rola (obrázok \ref{obr:eve_ng_pouzivatelia_web}).

Významom používateľských rolí v EVE-ng sa budeme zaoberať v kapitole \ref{chap:nasadenie_do_vyucovania} - \nameref{chap:nasadenie_do_vyucovania}.

Táto úprava je bližšie popísaná spolu s odchytenou prevádzkou v kapitole \ref{chap:cd} v bode \ref{item:spristupnenie_pouzivatelskych_roli}

\begin{figure}
    \centering
    \includegraphics[width=0.9\textwidth,trim={0cm 2cm 0cm 3.5cm},clip]{eve_ng_60040}
    \caption{Správa 60040 - úspešné odoslanie zoznamu používateľov zo servera}
    \label{obr:eve_ng_60040}
    
    \centering
    \includegraphics[width=1.0\textwidth,trim={0cm 5cm 0cm 4cm},clip]{eve_ng_pouzivatelia_web}
    \caption{Zoznam používateľov vo webovom rozhraní EVE-ng}
    \label{obr:eve_ng_pouzivatelia_web}
\end{figure}






\subsection{Úprava používateľských atribútov}

Atribúty jednotlivých používateľov je možné meniť na obrazovke \emph{User mangement} a môže ich meniť iba používateľ s administrátorskými oprávneniami t.j. s rolou \emph{admin}. Niektoré atribúty, ako sú napr. celé meno používateľa alebo email, sa síce dajú nastaviť, ale následne sa nedajú odstrániť, t.j. nastaviť na prázdnu hodnotu. Zmeny sa neprejavia ani vo web rozhraní v zozname používateľov
%(obrázok \ref{obr:eve_ng_pouzivatelia_web_email_predtym})
, ani v MySQL databáze.

\begin{figure}
    
\end{figure}

Skúsili sme teda odchytiť komunikáciu pri upravovaní spomenutých používateľských atribútov. Zistili sme, po úprave používateľa sa posiela správa \texttt{User saved (60042)} (obrázok \ref{obr:eve_ng_60042}).

\begin{comment}
\begin{figure}
    \centering
    \includegraphics[width=0.75\textwidth,trim={0cm 5cm 0cm 4cm},clip]{eve_ng_pouzivatelia_web_email_predtym}
    \caption{Stav pred odstránením e-mail atribútu používateľa \emph{ucitel}}
    \label{obr:eve_ng_pouzivatelia_web_email_predtym}
\end{figure}
\end{comment}

Nástroj \emph{grep} ukázal, že kód správy sa vyskytoval o.i. aj v súbore \\
\texttt{/opt/unetlab/html/includes/api\_uusers.php}, konkrétne aj vo funkcii \texttt{apiEditUUser}. Tá ziskava informácie o používateľovi z webového formulára pri úprave tohto používateľa a kontroluje ich správny formát. V prípade, že informácie zadané do webového formulára sú platné, aktualizujú sa atribúty pre konkrétneho používateľa v databáze, v opačnom prípade sa chybne zadané atribúty preskočia.

Problém bol v kontrole vstupov z webového formulára pri úprave používateľa, ktoré boli príliš striktné t.j. nedovoľovali zadať prázdnu hodnotu.

Riešenie spočívalo v upravení kritérii pre atrubúty tak, aby bol aj prázdny reťazec platnou hodnotou.

Po úprave sa už dalo používateľom nielen nastaviť ich celé meno či email, ale aj spomenuté atribúty odstrániť (obrázok \ref{obr:eve_ng_pouzivatelia_web_email_potom}).

Táto úprava je bližšie popísaná v kapitole \ref{chap:cd} v bodoch \ref{item:mail_meno_nejde_odstranit}-\ref{item:spristupnenie_pouzivatelskych_roli_pcap_posledny}

\begin{figure}
    \centering
    \includegraphics[width=0.75\textwidth,trim={0cm 5cm 0cm 2cm},clip]{eve_ng_60042}
    \caption{Správa 60042 - úspešné uloženie atribútov pre používateľa}
    \label{obr:eve_ng_60042}
    
    \centering
    \includegraphics[width=0.75\textwidth,trim={0cm 5cm 0cm 4cm},clip]{eve_ng_pouzivatelia_web_email_potom}
    \caption{Zoznam používateľov po odstránení e-mail atribútu pre používateľa \emph{ucitel}}
    \label{obr:eve_ng_pouzivatelia_web_email_potom}
\end{figure}






\subsection{Vypnutie správy o nízkom rozlíšení obrazovky}

Vypnutie tejto správy vypne chybové hlásenie o nízkom rozlíšení pri zmenšení okna prehliadača pod kritickú hranicu približne pod 992 pixelov. To umožní mať otvorený prehliadač s topológiou a napr.  Po zmenšení šírky okna pod túto hranicu sa zobrazila správa \\
\texttt{Display too small. This device is not large enough, you need 992px width at least.} (obrázok \ref{obr:eve_ng_display_too_small}).

\begin{figure}
    \centering
    \includegraphics[width=0.75\textwidth,trim={0cm 11cm 0cm 4cm},clip]{eve_ng_display_too_small}
    \caption{Chybová správa - \texttt{Display too small.}}
    \label{obr:eve_ng_display_too_small}
\end{figure}

Preto sme začali príkazom \emph{grep} hľadať súbory, obsahujúce časti tejto správy. Výstup príkazu obsahoval súbor  \texttt{/opt/unetlab/html/themes/default/index.html}, ktorý ako jediný obsahoval tento text v nižšie uvedenej časti kódu.

\renewcommand\baselinestretch{1}
\begin{Verbatim}[samepage=true]
...
<body>
        <div class="hidden-md hidden-lg container" id="small">
                ...
        </div>
        <div class="hidden-xs hidden-sm container-fluid" id="body">
                ...
        </div>
</body>
...
\end{Verbatim}
\renewcommand\baselinestretch{1.5}

Skúsili sme zakomentovať všetky riadky v sekcii \texttt{body}, ale následkom tejto zmeny sa stalo otváranie topológii nestabilné a vyskytovali sa rôzne grafické chyby vo vykresľovaní topológie a jej prvkov.

Po experimentovaní so zakomentovaním a upravovaním rôznych riadkov sme našli spôsob, ako túto správu vypnúť. Riešenie spočívalo v zakomentovaní celej sekcie \texttt{div} obsahujúcu atribút \texttt{id="small"} a odstránení tried \texttt{hidden-xs} a \texttt{hidden-sm} z definície sekcie \texttt{div} s atribútom \texttt{id="body"}.

\renewcommand\baselinestretch{1}
\begin{Verbatim}[samepage=true]
...
<body>
        <!--<div class="hidden-md hidden-lg container" id="small">
                ...
        </div>-->
        <!--<div class="hidden-xs hidden-sm container-fluid" id="body">-->
        <div class="container-fluid" id="body">
</body>
...
\end{Verbatim}
\renewcommand\baselinestretch{1.5}

Prvá úprava vypne hlásenie o nízkom rozlíšení obrazovky. Po uložení súboru po prvej úprave a znovunačítaní stránky uvidíme prazdnu bielu obrazovku, ak je okno prehliadača príliš malé t.j. menšie ako približne 992 pixelov.

Druhá úprava odstráni obmedzenie pri vykresľovaní obsahu topológie. Po uložení súboru po prvej úprave a znovunačítaní stránky uvidíme pôvodnú topológiu bez výrazných grafických chýb aj vtedy, ak je šírka okna prehliadača príliš malá.

Po vykonaných úpravách sa problém so zobrazovaním chybovej správy vyriešil, avšak sa vyskytol jeden kozmetický nedostatok. Po zmenšení okna sa zdeformoval posuvník na približovanie a odďaľovanie topológie po zmenšení okna pod kritickú hranicu šírky okna (obrázok \ref{obr:eve_ng_display_too_small_fixed_but_zoomslider_deformed}).

\afterpage{
\begin{figure}
    \centering
    \includegraphics[width=0.75\textwidth,trim={0cm 0cm 8cm 4cm},clip]{eve_ng_display_too_small_fixed_but_zoomslider_deformed}
    \caption{Vyriešenie problému s chybovým hlásením \texttt{Display too small.} a deformácia posuvníku na približovanie a odďaľovanie topológie}
    \label{obr:eve_ng_display_too_small_fixed_but_zoomslider_deformed}
\end{figure}
}

Tento problém sa nám nepodarilo ošetriť. Z \emph{Inšpektora prvkov} vo webovom prehliadači sme zistili, že tento jav môžu spôsobovať pevne zadané hodnoty v súboroch
\renewcommand\baselinestretch{1}
\begin{Verbatim}[samepage=true]
/opt/unetlab/html/themes/default/bootstrap/css/bootstrap.min.css
/opt/unetlab/html/themes/adminLTE/build/bootstrap-less/variables.less
\end{Verbatim}
\renewcommand\baselinestretch{1.5}

Samotný prvok sa volá \texttt{plus-minus-slider} a posuvná plocha sa volá \texttt{zoomslide}, Preto sa riešenie môže skrývať v úprave súborov z výstupu príkazov
\renewcommand\baselinestretch{1}
\begin{Verbatim}[samepage=true]
grep -rnw '/opt/unetlab/html/' -e 'plus-minus-slider'
grep -rnw '/opt/unetlab/html/' -e 'zoomslide'
\end{Verbatim}
\renewcommand\baselinestretch{1.5}

Bol to práve prvok \texttt{zoomslide}, ktorý sa neúmerne zväčšil. Jeho funkčnosť, približovať a oddaľovať prvky v topológii, však zostala zachovaná aj napriek tomuto vedľajšiemu účinku.

Táto úprava je bližšie popísaná v kapitole \ref{chap:cd} v bode \ref{item:display_too_small}






\subsection{Zatvorenie topológie so spustenými zariadeniami}
\label{chap:zatvorenie_topo_so_spustenymi_zariadeniami}

Topológiu sa nepodarí zatvoriť, pokiaľ obsahuje spustené zariadenia. Pri zatvorení topológie so spustenými zariadeniami sa vypíše chybové hlásenie \texttt{There are running nodes, you need to power off them before closing the lab.} (obrázok \ref{obr:eve_ng_running_nodes}).

\begin{figure}
    \centering
    \includegraphics[width=0.9\textwidth,trim={6cm 12cm 6cm 5cm},clip]{eve_ng_running_nodes}
    \caption{Chybové hlásenie - topológia so spustenými zariadeniami sa nedá zatvoriť}
    \label{obr:eve_ng_running_nodes}
\end{figure}

Preto sme nástrojom \emph{grep} hľadali, v ktorých súboroch sa vyskytujú časti tejto chybovej správy. Výstup príkazu ukazoval na súbory

\renewcommand\baselinestretch{1}
\begin{Verbatim}[samepage=true]
/opt/unetlab/html/themes/adminLTE/unl_data/js/angularjs/controllers/lab/labCtrl.js
/opt/unetlab/html/themes/default/js/messages_en.js
\end{Verbatim}
\renewcommand\baselinestretch{1.5}

Keďže v súbore \texttt{messages\_en.js} sa vyskytujú iba definície chybových hlásení, rozhodoli sme sa upravovať súbor \texttt{labCtrl.js}. V súbore

\renewcommand\baselinestretch{1}
\begin{Verbatim}[samepage=true]
/opt/unetlab/html/themes/adminLTE/unl_data/js/angularjs/controllers/lab/labCtrl.js
\end{Verbatim}
\renewcommand\baselinestretch{1.5}

sa síce táto správa vyskytuje, ale zakomentovanie ľubovoľnej relevantnej časti kódu v metóde \texttt{closeLab} nemá vplyv na funkčnosť t.j. chybové hlásenie sa pri zatvorení topológie napriek tomu zobrazí.

Preto sme sa nakoniec pozreli do súboru \texttt{messages\_en.js}. V ňom bola chybová správa definovaná v poli \texttt{MESSAGES} ako \texttt{MESSAGES[131]}. Znova sme začali hľadať výskyty tohto reťazca v súboroch nástrojom \emph{grep}. Výstup príkazu ukazoval na súbory

\renewcommand\baselinestretch{1}
\begin{Verbatim}[samepage=true]
/opt/unetlab/html/themes/default/js/functions.js
/opt/unetlab/html/themes/default/js/messages_en.js
\end{Verbatim}
\renewcommand\baselinestretch{1.5}

Keďže súborom \texttt{messages\_en.js} sme sa už zaoberali, pokračoval som súborom \texttt{functions.js}. V ňom sa vyskytovala aj funkcia \texttt{closeLab}. Tá obsahovala nielen chybové hlásenie, ale aj kontrolu, či v topológii sú už spustené zariadenia. Vypli sme teda túto kontrolu zakomentovaním riadku s podmienkou \texttt{if} a celej vetvy \texttt{else}.

\renewcommand\baselinestretch{1}
\begin{Verbatim}[samepage=true]
        //if (running_nodes == false) {
            ...
        //} else {
        //    deferred.reject(MESSAGES[131]);
        //}
\end{Verbatim}
\renewcommand\baselinestretch{1.5}

Potom sme sa odhlásili, vymazali vyrovnávaciu pamäť webového prehliadača a prihlásili sa do EVE-ng ako používateľ s rolou \texttt{admin}. Potom sme si otvorili súbor s topológiou a pridali do nej niekoľko zariadení. Spustili som zariadenie a pokúsil sa zatvoriť topológiu. Teraz sa chybové hlásenie nezobrazilo a topológia sa úspešne zatvorila. Po znovuotvorení rovnakej topológie zostali zariadenia spustené. Bolo možné aj spustiť ďalšie zariadenia.

Keď sme sa ešte predtým rozhodli riešiť problém so zatváraním topológie so spustenými zariadeniami, skúšali sme v súbore \texttt{functions.js} vo funkcii \texttt{closeLab} zakomentovať celý \texttt{for} cyklus v riadku

\renewcommand\baselinestretch{1}
\begin{Verbatim}[samepage=true]
    $.each(values, function (node_id, node) {
        if (node['status'] > 1) {
            running_nodes = true;
        }
    });
\end{Verbatim}
\renewcommand\baselinestretch{1.5}
keďže aj v ňom sa nastavovala premenná \texttt{running\_nodes} Po zakomentovaní cyklu sa topológia síce dala zatvoriť aj pri spustených zariadeniach, ale so zariadeniami v nej sa nedalo pracovať, napr. nebolo možné zastaviť už spustené zariadenia alebo spustiť ďalšie.

Po opravení tohto nedostatku sa ale vyskytol ďalší problém, ktorý sa odhalil až po vyriešení tohto, a síce, že po zatvorení jednej topológie 
%(obrázok \ref{obr:eve_ng_zle_portove_cisla_1}) 
a otvorení inej 
%(obrázok \ref{obr:eve_ng_zle_portove_cisla_2}) 
zariadenia vyzerali, že sú spustené, hoci predtým spustené neboli.
Otvorili sme jednu topológiu a spustili v nej zariadenia. Po zatvorení tejto topológie a otvorení inej ale zariadenia vyzerali ako spustené, aj keď predtým neboli spustené.
%Obidva 
%obrázky ukazujú 
%scenáre ukazovali rovnaký počet spustených zariadení v dvoch rôznych topológiách, pričom zariadenia boli spustené iba v prvej topológii
%na obrázku \ref{obr:eve_ng_zle_portove_cisla_1}. 
Zariadenia vo všetkých ďalších topológiách mali znefunkčnený vzdialený prístup. Buď sa na ne nedalo pripojiť vôbec, alebo sa konfigurovali zariadenia v pôvodnej topológii, čo dokazuje, že správne fungovali iba zariadenia v pôvodnej topológii. Pokiaľ mali topológie rovnaký počet zariadení a do druhej sme pridali nové zariadenie, toto zariadenie fungovalo bez komplikácii.

Tento jav nastal kvôli tomu, že portové čísla pre jednotlivé zariadenia v topológii sa začínajú číslovať od začiatku rozsahu, ktorý je pridelený danému používateľovi, bez ohľadu na to, ktorá topológia je momentálne otvorená.

Spomenutý problém s rovnakými portovými číslami pre zariadenia v rôznych topológiách sa nám nepodarilo vyriešiť, pretože sa jedná o hlbší problém, ktorého riešenie by znamenalo zmenu mechanizmu na prideľovanie portových čísel v EVE-ng. Spôsobu, akým EVE-ng generuje a prideľuje portové čísla zariadeniam je venovaná kapitola \ref{chap:priradovanie_portovych_cisel}.

\begin{comment}
\begin{figure}
    \centering
    \includegraphics[width=0.75\textwidth,trim={0cm 0cm 0cm 4cm},clip]{eve_ng_zle_portove_cisla_1}
    \caption{Prvá topológia a celkový počet spustených zariadení}
    \label{obr:eve_ng_zle_portove_cisla_1}
\end{figure}

\begin{figure}
    \centering
    \includegraphics[width=0.75\textwidth,trim={0cm 0cm 0cm 4cm},clip]{eve_ng_zle_portove_cisla_2}
    \caption{Druhá topológia a celkový počet spustených zariadení}
    \label{obr:eve_ng_zle_portove_cisla_2}
\end{figure}
\end{comment}

Komplikácie, na ktoré sme narazili počas nasadenia na predmety, sú popísané v kapitole \ref{chap:nasadenie_do_vyucovania} - \nameref{chap:nasadenie_do_vyucovania}.

Táto úprava je bližšie popísaná v kapitole \ref{chap:cd} v bode \ref{item:zatvorenie_topologie_so_spustenymi_zariadeniami}

Vo výsledku sme vyriešili iba problém zatvárania topolǵie so spustenými zariadeniami. Avšak ešte je potrebné vyriešiť priraďovanie portových čísel a vzdialený prístup k zariadeniam v následne otvorenej inej topológií.






\section{Administrácia}

EVE-ng server je vytvorený tak, aby po jeho konfigurácii bolo potrebné na ňom vykonávať minimálnu údržbu. V nasledujúcich častiach bude opísaný spôsob administrácie EVE-ng servera.





\subsection{Adresárova štruktúra}
\label{chap:adresarova_struktura}

Tu je uvedený krátky zoznam najdôležitejších adresárov v EVE-ng. Tie sú na EVE-ng serveri vytvorené pri inštalácii nástroja. Podrobnejší zoznam súborov a adresárov sa nachádza v kapitole \ref{chap:cd} v bode \ref{item:adresarova_struktura}, keďže zoznam adresárov je príliš rozsiahly.

\afterpage{
\begin{longtabu} to \textwidth {| X[3.0,l,m] | X[4.0,l,m] |}
\caption{Adresárová štrukúra EVE-ng servera}
\label{tab:adresare} \\
\hline
    \multicolumn{1}{|c|}{\textbf{Adresár}} & \multicolumn{1}{|c|}{\textbf{Popis}} \\
\hline
    \texttt{/opt/unetlab/addons/} & \makecell[lc]{Adresár obsahujúci všetky zariadenia, \\ ktoré je možné pridať do topológie. \\ Obsahuje podadresáre \texttt{dynamips}, \texttt{iol} a \texttt{qemu}, \\ podľa toho, pre aký typ hypervízora je zariadenie \\ určené - Dynamips, IOL alebo QEMU/KVM} \\
\hline
    \texttt{/opt/unetlab/html/} & \makecell[lc]{Adresár s webovou stránkou EVE-ng} \\
\hline
    \texttt{/opt/unetlab/html/templates/} & Šablóny pre každý typ zariadenia v topológii \\
\hline
    \texttt{/opt/unetlab/data/Logs} & Súbory o zázname činností na serveri \\
\hline
\end{longtabu}
}

Tento zoznam slúži iba na rýchlu orientáciu v súborovej štruktúre EVE-ng servera.




\subsection{Zálohovanie}
\label{chap:zalohovanie}

Kritické súbory, adresáre a databáza v EVE-ng sú zálohované pomocou vlastného skriptu (viď kap. \ref{chap:cd} bod \ref{item:backup_script}). Medzi zálohované súbory patria napr. konfiguračné súbory pre webový a SSH server, šablóny zariadení a zálohujú sa aj samotné všetky vytvorené skripty (viď kap. \ref{chap:cd} bod \ref{item:skripty})

Pri zálohovaní sa využívajú nástroje \emph{cron} a \emph{rsync}. Nástroj \emph{rsync}. Ten synchronizuje adresáre a súbory len vtedy, pokiaľ zistí, že sa majú nahradiť novšími verziami, čo je efektívny spôsob prenosu súborov, keďže sa prenášajú iba zmenené súbory. Nástroj \emph{cron} je nastavený tak, že vykonáva tento skript každý deň v noci, kedy sa na serveri vyskytuje minimálna aktivita.

Keďže zoznam zálohovaných prvkov v skripte je pomerne obsiahly, celý obsah skriptu je prítomný v kap. \ref{chap:cd} v bode \ref{item:backup_script}

Zálohovací server, resp. kontajner, je dostupný pod adresným rozsahom DMZ s posledným oktetom \emph{.45}. Ten obsahuje všetky potrebné súbory a adresáre potrebné na obnovenie EVE-ng servera v prípade jeho zlyhania.

Skript je schopný zálohovať aj súbory a virtuálne zariadenia z GNS3 servera. Na zálohovacom serveri sa zálohy z GNS3 servera objavia v samostatnom adresári.





\subsection{Monitorovanie}

Monitorovanie systému je dôležitým prostriedkom v prípade, že zaznamenáme nižšiu výkonnosť servera, alebo keď chceme vidieť jeho vyťaženie v reálnom čase. Na monitorovanie systémových zdrojov EVE-ng servera môžeme, okrem tradičného nástroja \emph{htop}, použiť aj vstavaný nástroj na monitorovanie systémových zdrojov vo webovom rozhraní EVE-ng. Monitorovaniu EVE-ng servera rôznymi nástrojmi sa venujeme v kapitole \ref{chap:cd} v bode \ref{item:monitorovanie}

Vstavaný monitorovací systém EVE-ng sa nachádza vo webovom rozhraní v časti \emph{System} -> \emph{System status}. Rovnaký panel je prístupný aj z rozhrania topológie v menu na ľavej strane obrazovky po kliknutí na položku \emph{Status}. Zobrazuje prehľad o aktuálnom percentuálnom vyťažení procesora, operačnej pamäte a diskového priestoru spolu s celkovým počtom spustených zariadení každého druhu. Nástroj je znázornený na obrázku \ref{obr:eve_ng_system_status}.

\begin{figure}
    \centering
    \includegraphics[width=0.75\textwidth,trim={0cm 0cm 0cm 4cm},clip]{eve_ng_system_status}
    \caption{Monitorovanie systému vo webovom rozhraní EVE-ng}
    \label{obr:eve_ng_system_status}
\end{figure}





\subsection{Správa používateľov EVE-ng}
\label{chap:sprava_pouzivatelov}

Používatelia webového rozhrania EVE-ng sú uložení v MySQL databáze na serveri. Zoznam používateľov webového rozhrania v EVE-ng je znázornený na obrázku \ref{obr:eve_ng_pouzivatelia_web_email_potom} (str. \pageref{obr:eve_ng_pouzivatelia_web_email_potom}) a je prístupný pod položkou \emph{Management} -> \emph{User management} z horného menu. Každý používateľ má definované nasledujúce stĺpce:

\begin{itemize}[noitemsep]
    \item \textbf{Username} - Používateľské meno. Používa sa na prihlásenie sa do webového rozhrania. Musí byť unikátne pre každého používateľa.
    \item \textbf{Email} - Emailová adresa.
    \item \textbf{Name} - Celé meno používateľa. Atribút má iba informatívny charakter.
    \item \textbf{Role} - Používateľská rola. Definuje oprávnenia pre používateľa.
    \item \textbf{POD} - Identifikačné číslo používateľa. Určuje rozsah portov, ktoré sa používajú na vzdialený prístup ku zariadeniam v topológii. Musí byť unikátne pre každého používateľa.
    \item \textbf{Actions} - Úprava atribútov používateľa (Edit) a odstránenie používateľa (Delete). Po kliknutí na tlačidlo \emph{Edit} v riadku vybraného používateľa sa otvorí dialógové okno na vytvorenie a úpravu používateľa zobrazené na obrázku \ref{obr:eve_ng_pouzivatelia_dialog} (str. \pageref{obr:eve_ng_pouzivatelia_dialog}).
\end{itemize}

Dialógové okno na vytvorenie a úpravu používateľa sa zobrazí po kliknutí na tlačidlá \emph{Add user} a \emph{Edit}. Pozostáva z týchto častí:

\begin{itemize}[noitemsep]
    \item User Name
    \item Password
    \item Password Confirmation
    \item Email
    \item Name
    \item Role
    \item POD
\end{itemize}

Všetky polia majú rovnaký význam ako v popise stĺpcov na obrazovke so zoznamom používateľov. Novým prvkom sú polia \emph{Password} a \emph{Password Confirmation}. Tie nie je nutné vypĺňať, ak ich nechceme meniť. Iba používatelia s rolou \emph{admin} môžu meniť heslá iným používateľom. Ak chceme používateľovi heslo zmeniť, je potrebné zadať nové heslo do obidvoch polí. Na zmenu hesla na nové nie je nutné zadávať pôvodné heslo. Pole \emph{User Name} sa pri úprave používateľa nedá zmeniť, dá sa iba jednorázovo nastaviť pri vytváraní používateľa. Pole \emph{POD} je vyplnené automaticky najnižším voľným identifikátorom.

Ak sme vykonali kroky v časti \ref{chap:eve_ng_pouzivatelske_role} - \nameref{chap:eve_ng_pouzivatelske_role}, budú po kliknutí na rozbaľovací zoznam pre atribút \emph{Role} dostupné, okrem role \emph{admin}, aj role \emph{editor} a \emph{user} (obr. \ref{obr:eve_ng_pouzivatelia_dialog}). Tieto role sa medzi sebou líšia oprávneniami na výkon určitých činností. Zoznam činností pre každú používateľskú rolu je popísaný v nižšie uvedených zoznamoch. \\

\noindent
Zoznam úloh, ktoré môže vykonávať používateľ s rolou \emph{user}:

\begin{itemize}[noitemsep]
    \item Prehliadať súbory a adresáre
    \item Otvoriť topológiu
    \item Spustiť a zastaviť zariadenia v topológii
\end{itemize}

\noindent
Zoznam úloh, ktoré môže vykonávať používateľ s rolou \emph{editor}:

\begin{itemize}[noitemsep]
    \item Všetko, čo môže vykonávať používateľ s rolou \emph{user}
    \item Spravovať súbory a adresáre - vytváranie, presúvanie, premenovanie, odstránenie
    \item Upravovať prvky v topológii - pridávanie, presúvanie, premenovanie, odstránenie
    \item Upravovať vybrané atribúty používateľov - meno, email
    \item Exportovať/importovať súbory s topológiami
    \item Zamknúť topológiu, aby prvky nebolo možné meniť a presúvať
\end{itemize}

\noindent
Zoznam úloh, ktoré môže vykonávať používateľ s rolou \emph{admin}:

\begin{itemize}[noitemsep]
    \item Všetko, čo môže vykonávať používateľ s rolou "editor"
    \item Zastaviť všetky zariadenia v "System -> Stop All Nodes"
    \item Zobraziť informácie o konkrétnom používateľovi cez API
    \item Spravovať všetkých používateľov - pridanie, upravenie, odstránenie
    \item Zapnúť/vypnúť UKSM v "System -> System status", ak je dostupné
    \item Zapnúť/vypnúť KSM v "System -> System status", ak je dostupné
    \item Zapnúť/vypnúť CPULimit v "System -> System status"
    \item Aktualizovať EVE-ng z web rozhrania cez koncový bod "/api/update" v UNetLab/EVE-ng API
\end{itemize}

Niektoré z týchto činností nie sú implementované vo webovom rozhraní EVE-ng. Činnosti, ktoré môžu vykonávať jednotlivé používateľské role sú definované v súbore \\
\texttt{/opt/unetlab/html/api.php}. Vyznačujú ich riadky

\renewcommand\baselinestretch{1}
\begin{Verbatim}[samepage=true]
  ...
  if (!in_array($user['role'], Array('admin'))) {
  ...

resp.

  ...
  if (!in_array($user['role'], Array('admin', 'editor'))) {
  ...
\end{Verbatim}
\renewcommand\baselinestretch{1.5}

Vyššie uvedené podmienky kontrolujú, či je používateľ s danou rolou oprávnený vykonať požadovanú operáciu, napr. vytvoriť používateľa, premenovať adresár, presunúť súbory do adresára a pod.

V prípade, že používateľ nemá dostatočné oprávnenia sa zobrazí chybové hlásenie \texttt{Not enough access privileges for this operation (90032)}, ktoré znázornené na obrázku \ref{obr:not_enough_access_privileges}.

\begin{figure}
    \centering
    \includegraphics[width=0.5\textwidth,trim={36cm 17cm 0cm 4cm},clip]{not_enough_access_privileges}
    \caption{Chybové hlásenie o nedostatočných oprávneniach používateľa}
    \label{obr:not_enough_access_privileges}
\end{figure}

Niekedy sa však táto správa pri vykonaní neoprávnenej činnosti nezobrazí, čo ale nemá žiadny vplyv na funkcionalitu a daná operácia sa nevykoná. Tento nedostatok je bližšie popísaný spolu s odchytením prevádzky v kapitole \ref{chap:cd} v bodoch \ref{item:nezobrazene_chybove_hlasenie} a \ref{item:nezobrazene_chybove_hlasenie_pcap}

Správa používateľov a MySQL databázy je podrobne vysvetlená vo vytvorenom návode v kapitole \ref{chap:cd} v bode \ref{item:sprava_pouzivatelov_a_databazy}