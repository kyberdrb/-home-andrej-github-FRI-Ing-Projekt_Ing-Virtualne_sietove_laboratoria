\chapter{Súčasný stav}

Je pomerne dobre známym faktom, že výučba nielen sieťových technológii je najúčinnejšia vtedy, keď má študent možnosť pracovať s vecami z reálneho sveta. Preto katedra disponuje fyzickými zariadeniami, ktoré pomáhajú študentom nabrať kvalitné skúsenosti. Avšak s fyzickými zariadeniami sa spájajú záväzky, ktoré nie je možné len tak ľahko prehliadnuť. Sú to napríklad:

\begin{enumerate}[noitemsep]
    \item Nedostatok prostriedkov na prevádzkovanie zariadení.
    \item Obmedzený prístup k zariadeniam. Ten je možný iba osobne v miestnosti špecializovanej na účel sieťového laboratória.
    \item Nedostatok priestoru pre fyzické zariadenia.
    \item Slabá miera izolácie pred prevádzkou generovanou v živej sieti.
    \item Postupná zastaranosť hardvéru alebo softvéru.
    \item Fyzické zariadenia sú náchylné na poruchy, čoho dôsledkom sú nesprávne fungujúce rozširujúce moduly alebo celé zariadenia. Tie treba buď opraviť, alebo vymeniť za nové. Následne treba myslieť na to, kam chybný hardvér umiestniť resp. ako ho odstrániť.
    \item Vyššia časová náročnosť pri prepájaní fyzických zariadení, predovšetkým pri náročnejších topológiách.
    \item So zvyšujúcim sa počtom zariadení v topológii rastie pravdepodobnosť, že ich študenti medzi sebou prepoja chybnými rozhraniami resp. sa pri prepájaní použije nesprávny typ kábla (použitie rovného Ethernet kábla namiesto kríženého a v.v.).
    \item Pomalšie spúšťanie a beh zariadenia, ktoré sú spôsobené načítaním operačného systému z pamäťovej karty a tým, že sieťové zariadenia sú špecializované na preposielanie rámcov a paketov, nie na rozbaľovanie komprimovaného súboru s operačným systémom.
    \item Náročné testovanie vzájomnej spolupráce zariadení. Ak by sme sa rozhodli vytvoriť topológiu so zariadeniami iných výrobcov, museli by sme si ich zaobstarať, čo vyžaduje ďalšie finančné a priestorové požiadavky.
\end{enumerate}

Vyššie uvedené problémy sa snažia riešiť rôzne virtualizačné platformy a nástroje, ktoré sú na nich postavené.

Vo svete pozorujeme trend rastúceho záujmu o virtuálne sieťové laboratóriá. Zo všetkých vymenujme dve univerzity, kde sa zaviedla táto forma výuky.

Na univerzite v \emph{Central University Taiwan} vytvorili v spolupráci s ďalšou univerzitou v Thajsku nástroj zvaný \emph{HVLab}, Hybrid Virtualization Laboratory. Ten v sebe integruje viacero prvkov. Používateľ pristupuje k topológii prostredníctvom webového rozhrania. Na serveri sa tieto topológie mapujú do GNS3 projektov. HVLab obsahuje aj tzv. logging, ktorý v reálnom čase zaznamenáva konfiguračné príkazy študenta a umožňuje učiteľovi vyhodnotiť jeho výkonnosť. Navyše nástroj obsahuje aj okno na výmenu správ v reálnom čase. Je rozdelené na skupinovú konverzáciu, ktorá má slúžiť na rýchlu výmenu konfigurácii medzi členmi tímu a súkromnú konverzáciu s učiteľom \cite{hvlab}. Nástroj je dostupný iba študentom na spomínanej univerzite.

Na \emph{Štátnej univerzite v Orenburgu} v Rusku sa skúmalo nasadenie ich vlastného návrhu virtuálneho sieťového laboratória na cloud platforme OpenNebula. Tak môžu poskytovať topológiu/infraštruktúru ako službu. Ich vlastný návrh riešenia spočíval v použití SDN návrhu v súčinnosti s nástrojmi Open vSwitch a OpenFlow. Topológie boli prístupné vo web rozhraní. Na grafickú interakciu s topológiou bol použitý komponent \emph{Draw2d touch}, ktorý umožňoval jednoduchú interaktivitu s prvkami topológie, ako napr. ich prepájanie čí presúvanie. Komponent na základe prepojení vygeneroval JSON súbor, ktorý topológiu definoval. Server pomocou tohto súboru vedel pracovať s topológiou a riadiť použité zdroje \cite{opennebula_lab}. Nástroj je dostupný iba pre študentov na spomínanej univerzite.

Tieto, a mnohé iné univerzity a školiace strediská si uvedomujú výhody virtualizácie sieťových prvkov a technológii pri vyučovaní.

Katedra sa tiež snaží držať krok so svetovým trendom. Aktívne sa na nej používajú viaceré riešenia virtualizovaného sieťového laboratória. Patria medzi ne Cisco Packet Tracer, Dynamips/Dynagen a GNS3.

Nástroj Cisco Packet Tracer sa momentálne používa na katedre pri vyučovaní kurzov CCNA Routing \& Switching, t.j. na predmetoch bakalárskeho stupňa štúdia Princípy informačno komunikačných technológii, Počítačové siete 1 a Počítačové siete 2.

Nástroj Dynamips/Dynagen sa používa pri výučbe predmetov bakalárskeho štúdia Počítačové siete 2, predmetov inžinierskeho štúdia Projektovanie sietí 1 a CCNP Routing (Pokročilé smerovanie v informačno-komunikačných sieťach).

Nástroj GNS3 sa na katedre používa lokálne, keďže ešte nie je dostatočne podrobne preskúmané nasadenie GNS3 ako vzdialený server. Slúži pre učiteľov na testovanie topológii a vyučovaných technológii nielen na predmetoch a kurzoch so zameraním na Cisco technológie, ale aj na testovanie spolupráce zariadení iných výrobcov, napr. spolupráca Cisco smerovačov s Juniper či Linux smerovačmi.

Katedra takisto disponuje aj nástrojom Cisco VIRL, ten sa však vo vyučovaní nepoužíva.

Katedre napriek tomu ešte stále chýba centralizované riešenie virtuálneho sieťového laboratória, ktoré by podporovalo všetky zariadenia spomenutých riešení. Túto situáciu sa v minulosti pokúšali niekoľkokrát zmeniť, pričom najbližšie sa zatiaľ dostal nástroj \emph{ViRo2} od Ing. Petra Hadača. Ten sa žiaľ pri vyučovaní používa veľmi zriedkavo.

Ďalšími vhodnými kandidátmi z prostredia open-source sú GNS3 a UNetLab, resp. EVE-ng. Hlavne GNS3 má už dlhoročnú tradíciu, narozdiel od už nevyvíjaného projektu UNetLab a jeho nasledovníka, EVE-ng. EVE-ng vyvíjala iná skupina vývojárov než UNetLab a dnes je už v štádiu, kedy dokáže robiť kvalitnú konkurenciu nástroju GNS3 a Cisco VIRL.

UNetLabv2, ktorý má byť tiež nasledovníkom projektu UNetLab, avšak z dielne pôvodného autora, žiaľ ešte nie je verejne prístupný, hoci sa na jeho vývoji pracuje.

Spomenuté nástroje sú podrobnejšie opísané v kapitole \ref{chap:nastroje_pre_siet_virt} - \nameref{chap:nastroje_pre_siet_virt}.